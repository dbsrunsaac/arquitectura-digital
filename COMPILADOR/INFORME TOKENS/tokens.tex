\documentclass[conference]{IEEEtran}
\IEEEoverridecommandlockouts
% The preceding line is only needed to identify funding in the first footnote. If that is unneeded, please comment it out.
\usepackage{cite}
\usepackage{amsmath,amssymb,amsfonts}
\usepackage{algorithmic}
\usepackage{graphicx}
\usepackage{textcomp}
\usepackage{xcolor}
\usepackage{tabularx}
\usepackage{multirow}
\usepackage{graphics} % for pdf, bitmapped graphics files
\usepackage{subfig}
\usepackage{subcaption}
\usepackage{hyperref}
\usepackage{academicons}
\usepackage{xcolor}
\def\BibTeX{{\rm B\kern-.05em{\sc i\kern-.025em b}\kern-.08em
		T\kern-.1667em\lower.7ex\hbox{E}\kern-.125emX}}
% Gráficas en MATLAB
\usepackage{tikz, pgfplots}
% Color Enlace
\definecolor{colorEnlace}{RGB}{0, 0, 0}
\hypersetup{
	colorlinks=true,
	linkcolor=colorEnlace,
	citecolor=colorEnlace,
	urlcolor=colorEnlace,
	pdfauthor={Ruth Juana Espino Puma},
	pdftitle={Introducción a LaTeX}
}
% Control 
\usepackage{amsmath}
\begin{document}
	
	\title{Generador Léxico - Tokenizador}
	\author{	
		\IEEEauthorblockN{Davis Bremdow Salazar Roa}
		\IEEEauthorblockA{Universidad Nacional de San Antonio Abad del Cusco}
		\textit{Escuela Profesional de Ingeniería Electrónica}\\
		\textit{Arquitectura de Microcontroladores y Microprocesadores}\\
		200353 \\\\
		Cusco, Perú
	}
	
	\maketitle
	
	\begin{abstract}
	\end{abstract}
	
	\begin{IEEEkeywords}
		Compilador, Generador de Léxico, Generador Sintáctico, Generador Sintáctico
	\end{IEEEkeywords}
	
	\section{Introducción}
		Los compiladores son herramientas esenciales en el desarrollo de software, ya que traducen el código fuente escrito en un lenguaje de programación de alto nivel a un lenguaje de bajo nivel o código máquina que puede ser entendido y ejecutado por la computadora. Esta conversión permite que los programas sean eficientes y optimizados para diferentes arquitecturas de hardware. Además, los compiladores ayudan a identificar errores en el código durante el proceso de traducción, mejorando así la calidad del software y facilitando el desarrollo de software.
	
	\section{Tokenizador de expresiones analíticas o ecuaciones}
	
	Una parte importante en el desarrollo de un compilador es la categorización del lenguaje de programación de alto a nivel en estructuras básicas mediante una etiqueta que las identifique, es por ello que un programa para este caso basado en la programación orientada a objetos sera de vital importancia para facilitar este proceso, brindando flexibilidad y escalamiento en el desarrollo del compilador.
	
	Para la creación del lenguaje de este programa se hizo uso del lenguaje CSharp y una de las metodologías de programación basada en la caracterización de objetos como entes virtuales, para este caso en concreto se definen 2 clases principales \textbf{Token} y \textbf{Lexer} las cuales se utilizaran para la creación del software de tokenización.
	
	Para la clasificación, subdivisión de caracteres para finalmente realizar la clasificación de cada token se definio una maquina de estados a nivel lógico y en la cual se definieron los estados necesarios para catalogar cada token en:
	\begin{itemize}
		\item Variable
		\item Asignación
		\item Operadores
	\end{itemize}
	
	Sin embargo para tener un mayor panorama en la creación de las expresiones será necesario agregar más tokens para el clasificamiento
	
		
	
	\bibliographystyle{IEEEtran}
	\bibliography{biblio}
\end{document}
